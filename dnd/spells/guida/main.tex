\documentclass[12pt]{article}
\usepackage[a4paper, portrait, margin=2cm]{geometry}    %paper shape
\usepackage[utf8]{inputenc}
\usepackage[T1]{fontenc}
\usepackage{hyperref}
\usepackage{amsfonts}
\usepackage{amsmath}
\usepackage{amssymb}

\title{Guida all'inserimento delle spell}
\author{Davide Borra}

\begin{document}
    \maketitle
    \section{Introduzione}
    Questo documento ha lo scopo di spiegare come inserire le spell nel database. Il software è disponibile al sito \url{davideborra.github.io}. 
    Attenzione: il sito non memorizza in automatico i dati per cui è necessario copiarli manualmente in un file di testo. L'aggiornamento della pagina comporta l'eliminaziona automatica dei dati. 

    \section{Inserimento}
    Di seguito si riportano alcune regole per l'inserimento delle spell:
    \begin{itemize}
        \item Il nome della spell deve essere scritto esattamente come sul manuale, rispettando anche le maiuscole e le minuscole.
        \item Per il livello il sito mostra sia un selettore che un numero. Il selettore serve solo a modificare il livello. Il numero è effettivamente il valore che viene inserito nel database. 0 indica un trucchetto.
        \item Per la scuola inserire solo il nome della scuola, senza la dicitura "Scuola di". Ad esempio un incantesimo di scuola di divinazione va inserito come "Divinazione".
        \item Nella dicitura tempo di lancio va inserita anche la possibilità di lancio come rituale, ad esempio "1 azione (rituale)"
        \item Le diciture gittata, componenti e durata sono le medesime presenti sul manuale.
        \item Il danno deve essere nella forma $(a\;-\!\!\!>b)dX + c \ \ (tipo)$ dove $a$ e $b$ sono rispettivamente è i numeri minimo e massimo di dadi, $X$ è il numero di facce, $c$ è il bonus al danno e $tipo$ è il tipo di danno. Inoltre se l'incantesimo richiede tiri salvezza va specificato "TS: tipo (dimezza/annulla)". Ad esempio
        \[\text{(3->10)d8 (tuono) TS: Costituzione (dimezza)}\]
    \end{itemize}

    \section{Consegna}
    I dati caricati vanno copiati in un file di testo e inviati a me. Il file deve essere nominato con il nome della scuola, ad esempio "\texttt{divinazione.json}".
\end{document}